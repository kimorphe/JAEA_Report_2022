\section{はじめに}
%なお,今回の研究で得られた成果を踏まえれば,今後,水和エネルギーモデルをより精緻化することで
%化学ポテンシャルを変化させながら膨潤量を計算することや, 所定の化学ポテンシャルにおいて生じる
%膨潤圧の計算が可能となる.これらの値は実測値と比較することができ,例えば,
%X線回折試験で得られた膨潤曲線や膨潤圧の計測値とシミュレーション結果を比較すれば,
%層間イオンの種類や組成に応じた膨潤挙動の解釈や推定にも利用できると期待される.
%CGMD法では,このような膨潤解析を任意の温度や乾燥密度,飽和度で行うことができるため,
%その信頼性が担保されれば,実験が困難な温度や密度,湿度条件での膨潤圧や膨潤および水分量
%を推定することにも役立つ.このようなシミュレーションが実現すれば,ベントナイト緩衝材の
%再冠水時の浸透や膨潤挙動を調べる上で有用な知見与えることが期待できる.
本共同研究では,粘土含水系の組織構造形成シミュレーションを目的とした
粗視化分子動力学法(Coarse-Grained Molecular Dynamics: CG-MD)の開発に
取り組んできた.CG-MD法では,分子動力学法と同様に,粒子多体系の
運動方程式を解くことで粘土分子の位置と変形を求める.ただし,対象とする粒子は
粘土分子を構成する原子そのものではなく,粘土分子の単位構造を一つの粒子に粗視化
して扱う.また,粘土層間の交換性陽イオンへ水和することによって保持される水分の
量と位置を,モンテカルロ法を用いて決定する.これら分子の運動と水分の配置を交互
に計算しながら,最終的には平衡状態における粘土と層間水の配置(組織構造)を求める.\\

CG-MD法で層間水配置を求める際のモンテカルロ計算を行うために,
水分量に応じて変化する水和エネルギーを用いる.粘土の膨潤挙動は
交換性陽イオンの種類によって大きく異なるるため,このことは,
現実の粘土をモデル化するためにはイオン種に応じた
水和エネルギーを設定することが必要であることを意味する.
一方,昨年度の研究では,シミュレーションモデルの膨潤状態が
離散的に推移するように設計した水和エネルギーを用いていた.
これは,仮想的な粘土を想定したものであるため,現実を粘土を想定した
水和エネルギーの設定方法を考案することが課題として残されてていた。
また,昨年度までの研究では,化学ポテンシャルを指定したCG-MDシミュレーションを行っていた。
化学ポテンシャルは,実験条件として直接指定できる量ではない.
そのため,CG-MDシミュレーションと実験の対比が難しく,この点も解決すべき
課題の一つとして残されていた.\\

以上を踏まえ,本年度は次の2つの課題に取り組む.
\begin{enumerate}
\item
	実験で観測された膨潤挙動を踏まえた水和エネルギーモデルの開発
\item
	相対湿度を指定したCG-MDシミュレーションの実現
\end{enumerate}
水和エネルギーモデルの開発は,ベントナイト緩衝材の材料として重要となる
Na型モンモリロナイトを対象とする.また,膨潤挙動に関する実験は,
その場X線回折試験により相対湿度と層間距離の関係がこれまでに詳しく
調べられていることから,その結果を利用した水和エネルギーのモデル化を
行う.相対湿度は,熱力学理論式を用いて化学ポテンシャルに変換することが
できるため,指定された相対湿度を化学ポテンシャルに換算してCG-MD計算
に与えれば,二番目の課題は解決することができる.
ただしこのとき,水和エネルギーから与えられる化学ポテンシャルと,
相対湿度から指定された化学ポテンシャルが,互いに矛盾するもので
あってはならない.本研究では,CG-MD系と環境の湿度が平衡状態で
互いに一致することを条件として水和エネルギーをモデル化する.
ただしそのためには,本文中で詳しく述べるように,
X線回折試験による膨潤観察結果だけでなく,水分量と層間距離の関係を与える必要がある.
後者の関係は実験で得られないため,全原子MD計算で得られる関係を用いる.
今年度は,以上の方法で作成した新しい水和エネルギーモデルを用い,相対湿度をした
組織構造形成のCG-MDシミュレーションを試行する.\\


以下では,化学ポテンシャルと相対湿度の関係を示した後,CG-MD系が有する
自由エネルギーの一部として水和エネルギーを導入する.次に,
CG-MD法の自由エネルギーから与えられる化学ポテンシャルと環境の化学
ポテンシャルの平衡条件を示す.この平衡条件を表す関係を,
モンモリロナイトの粉末X線回折試験と全原子MD計算による膨潤曲線を用いて
水和エネルギーについて解くことで,Na型モンモリロナイトの水和エネルギー
モデルを得る.このようにして求めた水和エネルギーモデルを用い,
相対湿度を指定したCG-MD計算を行う.CG-MD計算は,6つの異なる相対湿度で行い,
湿度に応じた層間距離や組織構造の変化について調べ,Na型モンモリロナイトの
X線回折試験結果と整合する膨潤挙動がCG-MDシミュレーションに現れる
ことを示す.最後に,本年度の研究成果についてまとめと今後の課題を述べる.

%成果の意義:
%本年度の研究の結果、実験で制御することのできるマクロ変数である、
%温度、圧力、体積、水分量と湿度、全てをモデル上、直接設定できるようになった。
%このことは、実験に対応させた条件でCG-MDシミュレーションを行い、 
%実験結果と比較が可能となったことを意味する。 
%例えば,実験で観測されたXRDパターンをCG-MD法で得たモデルから合成すれば、
%両者を直接比較することができ、モデルの良し悪しを議論することができる。
%モデルの妥当性がこのような比較で今後検証できれば、逆に、XRDパターンの より詳細な分析が可能になり、
%XRDパターンを実験と計算で比較することで どのような組織構造が現実に出現しているかを探ることができる。
%現状では、粘土含水系の組織構造を観測する方法がないため、このように 実測と対照が可能な組織構造モデル
%を生成するシミュレーション技術は 有用である。また、現在は、水分分布は平衡状態でのみシミュレーション可能
%だが、非平衡状態から平衡状態への緩和挙動を調べることで、
%水分拡散係数を算出できる可能性がある。このような数値解析技術を実現することができれば、
%メソスケールでの不飽和水分浸透挙動の解析を行う道が拓ける。 このことは、粘土含水系の性質を原子、
%分子スケールでの挙動から理解するという学問的な視点のみならず、ベントナイト緩衝材の再冠水時の挙動を
%評価、予測する上でも有用な知見を与えることができるという意味で意義をもつと言える.

