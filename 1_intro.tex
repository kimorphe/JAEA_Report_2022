\section{はじめに}
本研究では,粘土含水系の組織構造形成シミュレーションを目的とした
粗視化分子動力学法(Coarse-Grained Molecular Dynamics: CG-MD)の開発に
取り組んできた.CG-MD法は,分子動力学法と同様,粒子多体系の運動方程式を解くことで
粘土分子の位置と変形を求める.ただし,対象とする粒子は粘土分子を構成する原子そのものでなく,
粘土分子の単位構造を一つの粒子に粗視化して扱う.また,粘土層間
に保持される水分の量と位置をモンテカルロ法で決定する.
これら粗視化粒子の運動計算と水分配置計算を交互に行い,最終的には平衡状態における
粘土分子と層間水の配置を求める.
粘土分子の集合系を扱ったシミュレーションはCG-MD法以外でも行われている\cite{Eb2014}-\cite{Katti}.
しかしながら,それらは粘土鉱物を剛体分子として扱っているため,
分子の屈曲やその結果として生じる間隙を表現することはできない.
また,層間水を扱うメカニズムを備えていないため,水分配置や膨潤を扱うこともできない.
本研究のCG-MD法は,分子変形と層間水を考慮することができる点において
従来法と異なる.
\\
\hspace{\parindent}
CG-MD法で層間水配置を求めるモンテカルロ計算では,
水和エネルギーを層間水量の関数として与える必要があり,
モデル化された粘土の膨潤挙動は水和エネルギーの与え方
によって変化する.
実験で観測される粘土の膨潤挙動は,交換性陽イオンの種類によって
異なるため,特定の粘土をモデル化するにはイオン種に応じた
水和エネルギーの設定が必要になる.
この点に関して昨年度の研究では,シミュレーションモデルの膨潤状態が
離散的に推移するように設計した水和エネルギーを用いていた.
これは,仮想的な粘土を想定したもので,実験結果に基づく
水和エネルギーの設定方法を考案することが課題として残されていた.
また,水分量の変化を伴うCG-MD解析では,化学ポテンシャルを指定した
シミュレーションは可能であった.しかしながら,
化学ポテンシャルは実験時に直接指定できる量ではなく,
このことがCG-MDシミュレーションと実験の対比を難しくする原因の一つであった.\\
\hspace{\parindent}
以上を踏まえ,本年度は次の2つの課題を解決することを目的として研究を行う.
\begin{itemize}
\item
	実験結果に基づく水和エネルギーのモデル化方法の提案
\item
	相対湿度を指定したCG-MDシミュレーションの実現
\end{itemize}
第一の課題である水和エネルギーのモデル化では,Na型モンモリロナイトを対象とする.
そのために,X線回折試験で明らかにされている相対湿度と層間距離の関係を利用する.
第二の課題に関しては,熱力学理論式を用い,相対湿度を化学ポテンシャルに変換して
CG-MD計算の条件として与えることで,
相対湿度を指定したシミュレーションを実現する.
ただし,水和エネルギーモデルが規定する粘土含水系の化学ポテンシャルと,
相対湿度から換算された化学ポテンシャルが矛盾するものであってはならない.
そこで,CG-MD系と環境の化学ポテンシャルが,任意の平衡状態で一致することを
利用して水和エネルギーを決定する.
そのためには,相対湿度と層間距離の関係に加え,層間水量と層間距離の関係
が必要になある.前者は実験で得られるが,後者は実験で得られないため,
全原子分子動力学による計算結果を用いる.以上の方法で作成した新しい水和
エネルギーモデルを用い,相対湿度が一定のもとでどのような組織構造
が得られるか,CG-MD法でのシミュレーションを行う.\\
\hspace{\parindent}
本稿,以下の節では,はじめに化学ポテンシャルと相対湿度の関係を示した後,
CG-MD系がもつ自由エネルギーの一部として水和エネルギーを導入する.
次に,水和エネルギーに由来する化学ポテンシャルと環境の化学ポテンシャルの
平衡条件を示す.この条件を,X線回折試験と全原子MD計算結果を
踏まえて水和エネルギーについて解き,Na型モンモリロナイトの
水和エネルギーモデルを決める.続いて,新たに得られた水和エネルギーモデル
を用い,相対湿度を指定したCG-MD計算を行う.CG-MD計算では,
湿度に応じた層間距離や組織構造の変化を調べ,
X線回折試験結果に見られるNa型モンモリロナイトの特徴を反映した
膨潤挙動がシミュレーション結果に現れることを示す.
最後に,本年度の研究のまとめと今後の課題を述べる.
%なお,今回の研究で得られた成果を踏まえれば,今後,水和エネルギーモデルをより精緻化することで
%化学ポテンシャルを変化させながら膨潤量を計算することや, 所定の化学ポテンシャルにおいて生じる
%膨潤圧の計算が可能となる.これらの値は実測値と比較することができ,例えば,X線回折試験
%で得られた膨潤曲線や膨潤圧の計測値とシミュレーション結果を比較すれば,
%層間イオンの種類や組成に応じた膨潤挙動の解釈や推定にも利用できると期待される.
%CGMD法では,このような膨潤解析を任意の温度や乾燥密度,飽和度で行うことができるため,
%その信頼性が担保されれば,実験が困難な温度や密度,湿度条件での膨潤圧や膨潤および水分量
%を推定することにも役立つ.このようなシミュレーションが実現すれば,ベントナイト緩衝材の
%再冠水時の浸透や膨潤挙動を調べる上で有用な知見与えることが期待できる.
