\section{はじめに}
本研究では,粘土含水系の組織構造形成シミュレーションを目的とした
粗視化分子動力学法(Coarse-Grained Molecular Dynamics: CG-MD)の開発に
取り組んできた.CG-MD法は,分子動力学法と同様に,粒子多体系の運動方程式を解くことで
粘土分子の位置と変形を求める.ただし,対象とする粒子は粘土分子を構成する原子そのものでなく,
粘土分子の単位構造を一つの粒子に粗視化して扱う.また,粘土層間の交換性陽イオンへの水和
によって保持される水分の量と位置をモンテカルロ法で決定する.
これら粗視化粒子の運動と水分配置の計算を交互に行い,最終的には平衡状態における粘土と
層間水の配置(組織構造)を求める.
粘土分子の集合系を扱ったシミュレーションはこれまでにも行われている\cite{Eb2014}-\cite{Katti}.
しかしながら,それらはいずれも粘土鉱物を剛体分子として扱っているため,
分子の屈曲やその結果として生じる間隙を表現することはできない.
また,層間水を扱うメカニズムを備えていないため,水分配置や膨潤を扱うこともできない.
本研究のCG−MD法は,分子変形と層間水を考慮することができる点において、
従来法とは明確に異なっている.
\\
\hspace{\parindent}
CG-MD法で層間水の配置を求めるモンテカルロ計算では,
水和エネルギーを水分量の関数として与える必要があり,
モデル化された粘土の膨潤挙動は水和エネルギーの与え方
によって変化する.
実験で観測される粘土の膨潤挙動は,交換性陽イオンの種類によって
異なるるため,特定の粘土をモデル化するにはイオン種に応じた
水和エネルギーの設定が必要になる.
これに対して,昨年度の研究では,シミュレーションモデルの膨潤状態が
離散的に推移するように設計した水和エネルギーを用いていた.
これは,仮想的な粘土を想定したものであるため,実験結果に基づく
水和エネルギーの設定方法を考案することが課題として残されてていた。
また,水分量の変化を伴うCG-MD解析では,化学ポテンシャルを指定した
シミュレーションをまでが可能であった.しかしながら,
化学ポテンシャルは実験条件として直接指定できる量ではなく,
このことがCG-MDシミュレーションと実験の対比を困難にする一つの
原因となっていた.\\
\hspace{\parindent}
以上のことを踏まえ,本年度は次の2つの課題を解決することを目的として研究を行った.
\begin{itemize}
\item
	実験結果に基づく水和エネルギーのモデル化方法の提案
\item
	相対湿度を指定したCG-MDシミュレーションの実現
\end{itemize}
水和エネルギーのモデル化は,ベントナイト緩衝材の材料として重要となる
Na型モンモリロナイトを対象として行う.その際,X線回折試験で詳細に調べられている,
相対湿度と層間距離の関係を利用する.2つめの課題に関しては,熱力学理論式を用いて
相対湿度を化学ポテンシャルに変換してCG-MD計算の条件として与えることで,
相対湿度を指定したシミュレーションを実現する.
ただしこのとき,水和エネルギーモデルが与える粘土含水系の化学ポテンシャルと,
相対湿度から指定された化学ポテンシャルが矛盾するものであってはならない.
そこで,CG-MD系と環境の相対湿度が任意の平衡状態で一致することを
条件として水和エネルギーを与える.
そのためには,相対湿度と層間距離の関係に加え,水分量と層間距離の関係
を与える必要がある.後者の関係は実験で得られないため,全原子分子動力学
計算で得られる関係を用いる.以上の方法で作成した新しい水和エネルギーモデルを用い,
相対湿度が一定のもとでどのような組織構造が得られるか,CG-MD
シミュレーションを行う.\\
\hspace{\parindent}
本稿の以下に続く節では,化学ポテンシャルと相対湿度の関係を示した後,
CG-MD系がもつ自由エネルギーの一部として水和エネルギーを導入する.
次に,水和エネルギーに由来する化学ポテンシャルと環境の化学ポテンシャルの
平衡条件を示す.この条件を,X線回折試験と全原子MD計算による膨潤曲線を
踏まえて水和エネルギーについて解き,Na型モンモリロナイトの
水和エネルギーモデルを得る.続いて,新たに得られた水和エネルギーモデル
を用い,相対湿度を指定したCG-MD計算を行う.CG-MD計算では,
湿度に応じた層間距離や組織構造の変化を調べ,Na型モンモリロナイトの
X線回折試験結果を反映したる膨潤挙動がシミュレーション結果に現れる
ことを示す.最後に,本年度の研究のまとめと今後の課題を述べる.
%なお,今回の研究で得られた成果を踏まえれば,今後,水和エネルギーモデルをより精緻化することで
%化学ポテンシャルを変化させながら膨潤量を計算することや, 所定の化学ポテンシャルにおいて生じる
%膨潤圧の計算が可能となる.これらの値は実測値と比較することができ,例えば,X線回折試験
%で得られた膨潤曲線や膨潤圧の計測値とシミュレーション結果を比較すれば,
%層間イオンの種類や組成に応じた膨潤挙動の解釈や推定にも利用できると期待される.
%CGMD法では,このような膨潤解析を任意の温度や乾燥密度,飽和度で行うことができるため,
%その信頼性が担保されれば,実験が困難な温度や密度,湿度条件での膨潤圧や膨潤および水分量
%を推定することにも役立つ.このようなシミュレーションが実現すれば,ベントナイト緩衝材の
%再冠水時の浸透や膨潤挙動を調べる上で有用な知見与えることが期待できる.
