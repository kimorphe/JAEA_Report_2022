\section{はじめに}
本共同研究では,これまで粗視化分子動力学法(CGMD法)のプログラムを開発し,
粘土含水系の組織構造形成に関するシミュレーションを行ってきた.
昨年度の研究では,粘土含水系が環境(外界)との間で水分をやり取りして
吸排水が生じるシミュレーションを行うために,化学ポテンシャル一定条件での
計算が可能な形へプログラムを拡張した.
化学ポテンシャルはその逆数が湿度の高低に相当するパラメータとして物理的意味を持つ.
従って,設定した化学ポテンシャルが高い(環境の湿度が低い)ときに排水を,
低い(環境の湿度が高い)ときには吸水を起こす方向へ粘土含水系の状態は変化する.
その結果,体積が拘束されている場合は吸水(排水)によって膨潤圧力が上昇(低下)する.
体積が拘束されていない場合は吸水によって体積膨張を,排水によって体積収縮を生じる.
\\
本共同研究では,粘土含水系の組織構造シミュレーションを行うための粗視化分子動力学法(Coarse-Grained Molecular Dynamics: CG-MD)を開発してきた.CG-MD法では,計算モデルである粘土含水系への水分の出入りを制御するために,層間水量に応じて変化する水和エネルギーを与える必要がある.昨年度の研究では,シミュレーションモデルが離散的な膨潤状態の推移を示すよう,仮想的な粘土の水和エネルギーモデルを与えて計算を行っていた.これに対して本年度は,Na型モンモリロナイトのX線回折試験で実際に観測された膨潤挙動を反映した水和エネルギーモデルを作成し,吸水膨潤のCG-MDシミュレーションを行った.加えて,これまで化学ポテンシャルを指定した計算までが可能であったが,新たに相対湿度を指定した計算を実施することができるよう手法の改良を行った.
\\
%
このような吸排水の原動力は粘土分子表面への水和に関するエネルギーにある.
水分子は電荷を帯びた粘土表面に水和することでよりエネルギーの低い安定な状態になる.
水和に起因したこのエネルギー変化を水和エネルギーと呼ぶ.
CGMD法では水和エネルギーと水和水量の関係を仮定し,モンテカルロ法
よる緩和計算を経て系の平衡状態における水分量と配置を決定することができる.
%
この方法により昨年度の研究では,水和エネルギーが水和量の増加に対して
単調減少すると仮定して計算を行った.その結果,吸排水や膨潤圧が,
化学ポテンシャルに応じて期待したように生じることが確認できた.
しかしながら,単調減少する水和エネルギー関数は,一つのパラメータで
規定される単純なもので,層間イオン種に応じたモンモリロナイトの複雑な
膨潤挙動を再現するためには十分でない.
例えばNa型モンモリロナイトでは,相対湿度に対して,ステップ状の膨潤を
示すことがX線観測の結果として知られている.
このように離散的な膨潤状態を取る挙動は,単調な水和エネルギー関数からは生じ得ず,
モンモリロナイトの膨潤挙動を表現するためには,いくつかの極値をもつような非単調な
水和エネルギーモデルが必要であることを意味する.そのような水和エネルギーモデル
を開発するには,水和エネルギーの局所的な変動が,膨潤挙動にどのような影響を与える
かについて十分理解することが必要となる.そこで本年度は,水分量に対して単調に増加する
昨年度までのモデルに振動成分を加え,水和エネルギーの局所的変動が膨潤に与える
影響を調べた.その結果,新しい水和エネルギーモデルを用いた計算では,
水和エネルギーの極大点を避けるように膨潤状態が決まり,
化学ポテンシャルの変化に対して極大点付近では膨潤状態が離散的に推移することを示す.


以下では,CGMD法の基本的な考え方をはじめに述べる.
特にCGMDモデルにおける水和水量の表現について要点を述べる.
次に,今回の解析に用いた水和エネルギーモデルの詳細とその意図を説明する.
続いて,温度,体積,化学ポテンシャル一定の条件で行った緩和
シミュレーションの結果を示す.その際,どのような膨潤状態が
支配的であるかを見るために,水和水層厚の頻度分布を示す.
この頻度分布から膨潤状態が化学ポテンシャルに対して必ずしも連続的に変化せず,
水和エネルギーの極大点を避けるように水和状態が選択されることを示す.
最後に,本年度の研究結果についてまとめ今後の課題を示す.
%	CGMD法の将来性,利用方法について
%なお,今回の研究で得られた成果を踏まえれば,今後,水和エネルギーモデルをより精緻化することで
%化学ポテンシャルを変化させながら膨潤量を計算することや, 所定の化学ポテンシャルにおいて生じる
%膨潤圧の計算が可能となる.これらの値は実測値と比較することができ,例えば,
%X線回折試験で得られた膨潤曲線や膨潤圧の計測値とシミュレーション結果を比較すれば,
%層間イオンの種類や組成に応じた膨潤挙動の解釈や推定にも利用できると期待される.
%CGMD法では,このような膨潤解析を任意の温度や乾燥密度,飽和度で行うことができるため,
%その信頼性が担保されれば,実験が困難な温度や密度,湿度条件での膨潤圧や膨潤および水分量
%を推定することにも役立つ.このようなシミュレーションが実現すれば,ベントナイト緩衝材の
%再冠水時の浸透や膨潤挙動を調べる上で有用な知見与えることが期待できる.

\section{追加テキスト}

    昨年度(まで)の研究:メソスケール粗視化分子動力学法の開発。 ターゲットは粘土含水系.緩衝材の重要な成分であるNa型モンモリロナイトの 組織構造をモデル化するためのシミュレーション法開発. 水分の出入りを考慮した粘土含水系の挙動が調べられるようになっていた。 水分の出入りは水和エネルギー関数に従って決定される。水分画の増減で膨張や収縮がおきる。 粘土粒子の位置の変更、つまり運動と、水分の移動の両方を同時に扱う必要がある。 前者はMD法、後者はMC法を使い、2つの組み合わせでこのような解析を実現してきた。 その際,粘土の膨潤挙動は層間の交換性カチオンによって大きくことなる。 このような給排水、膨潤の個性をモデルに反映するには、イオン種に応じた 水和エネルギーモデルを設定する必要がある。 昨年度は仮想的に設定した水和エネルギーモデルを用いていたため、 これを現実的なものとすることが課題として残されていた。
    具体的な課題: そこで本年度は、Na型モンモリロナイトの膨潤挙動を模擬するための水和エネルギーモデルの開発に取り組む。 そのためには、実験で観察されている膨潤挙動をモデルに反映する方法を考案する必要がある。 その際問題となるのは、実験では粘土試料がおかれた環境の温度や相対湿度を変化させたときの 膨潤挙動が観測されている。一方、これまでのCG-MDモデルでは、化学ポテンシャルを指定した計算 が可能なものの、相対湿度を条件として与えることはできなかった。 そのために、昨年度までの方法では実験条件に対応させたCG-MD計算を行うことができなかった。
    課題に対するアプローチ: 以上の課題に対し、今年度は、熱力学関係式を用いて相対湿度を化学ポテンシャルに変換し, 相対湿度を指定したCG-MDシミュレーションを行う. 同時に,実験で与えられた膨潤曲線を踏まえた、水和エネルギー関数を設定する方法を 考案し,新しい水和エネルギー関数を用いた組織構造形成シミュレーションを試行する.
    成果の概要: XRD測定で得られた膨潤曲線と、全原子MD計算で得られた膨潤曲線を組み合わせ て用いることで、実際のNa型モンモリロナイトの挙動を考慮することのできる 水和ネルギー関数モデルを作成した。 また、CG-MD系が置かれた環境の湿度を指定し、それを熱力学関係式を用いて 化学ポテンシャルに変換して用いることで、湿度一定下でのCG-MDシミュレーション が可能となった。これらの改良を経たシミュレーションで、 いくつかの相対湿度で組織構造シミュレーションを行った結果, 実験で観測されやすい膨潤状態が選択的に出現することが明らかにされた。
    成果の意義: 本年度の研究の結果、実験で制御することのできるマクロ変数である、 温度、圧力、体積、水分量と湿度、全てをモデル上、直接設定できるようになった。 このことは、実験に対応させた条件でCG-MDシミュレーションを行い、 実験結果と比較が可能となったことを意味する。 例えば,実験で観測されたXRDパターンをCG-MD法で得たモデルから合成すれば、  両者を直接比較することができ、モデルの良し悪しを議論することができる。 モデルの妥当性がこのような比較で今後検証できれば、逆に、XRDパターンの より詳細な分析が可能になり、XRDパターンを実験と計算で比較することで どのような組織構造が現実に出現しているかを探ることができる。 現状では、粘土含水系の組織構造を観測する方法がないため、このように 実測と対照が可能な組織構造モデルを生成するシミュレーション技術は 有用である。また、現在は、水分分布は平衡状態でのみシミュレーション可能 だが、非平衡状態から平衡状態への緩和挙動を調べることで、水分拡散係数を 算出できる可能性がある。このような数値解析技術を実現することができれば、 メソスケールでの不飽和水分浸透挙動の解析を行う道が拓ける。 このことは、粘土含水系の性質を原子、分子スケールでの挙動から理解する という学問的な視点のみならず、ベントナイト緩衝材の再冠水時の挙動を 評価、予測する上でも有用な知見を与えることができるという意味で、 重要な意義のあることと言える.

